%%% Template originaly created by Karol Kozioł (mail@karol-koziol.net) and modified for ShareLaTeX use

\documentclass[a4paper,11pt]{article}

\usepackage[T1]{fontenc}
\usepackage[utf8]{inputenc}
\usepackage{graphicx}
\usepackage{xcolor}

\renewcommand\familydefault{\sfdefault}
\usepackage{tgheros}
\usepackage[defaultmono]{droidmono}

\usepackage{amsmath,amssymb,amsthm,textcomp}
\usepackage{enumerate}
\usepackage{multicol}
\usepackage{tikz}

\usepackage{geometry}
\geometry{left=25mm,right=25mm,%
bindingoffset=0mm, top=20mm,bottom=20mm}


\linespread{1.3}

\newcommand{\linia}{\rule{\linewidth}{0.5pt}}

% custom theorems if needed
\newtheoremstyle{mytheor}
    {1ex}{1ex}{\normalfont}{0pt}{\scshape}{.}{1ex}
    {{\thmname{#1 }}{\thmnumber{#2}}{\thmnote{ (#3)}}}

\theoremstyle{mytheor}
\newtheorem{defi}{Definition}

% my own titles
\makeatletter
\renewcommand{\maketitle}{
\begin{center}
\vspace{2ex}
{\huge \textsc{\@title}}
\vspace{1ex}
\\
\linia\\
\@author \hfill \@date
\vspace{4ex}
\end{center}
}
\makeatother
%%%

% custom footers and headers
\usepackage{fancyhdr}
\pagestyle{fancy}
\lhead{}
\chead{}
\rhead{}
\lfoot{Sentiment Analysis}
\cfoot{}
\rfoot{Page \thepage}
\renewcommand{\headrulewidth}{0pt}
\renewcommand{\footrulewidth}{0pt}
%

% code listing settings
\usepackage{listings}
\lstset{
    language=Python,
    basicstyle=\ttfamily\small,
    aboveskip={1.0\baselineskip},
    belowskip={1.0\baselineskip},
    columns=fixed,
    extendedchars=true,
    breaklines=true,
    tabsize=4,
    prebreak=\raisebox{0ex}[0ex][0ex]{\ensuremath{\hookleftarrow}},
    frame=lines,
    showtabs=false,
    showspaces=false,
    showstringspaces=false,
    keywordstyle=\color[rgb]{0.627,0.126,0.941},
    commentstyle=\color[rgb]{0.133,0.545,0.133},
    stringstyle=\color[rgb]{01,0,0},
    numbers=left,
    numberstyle=\small,
    stepnumber=1,
    numbersep=10pt,
    captionpos=t,
    escapeinside={\%*}{*)}
}

%%%----------%%%----------%%%----------%%%----------%%%

\begin{document}

\title{Sentiment Analysis}

\author{Tanishq Chahal, McMaster University}

\date{05/04/2024}

\maketitle

\section*{1) Functions header}

The header file defines functions and a structure for handling data related to sentiment analysis. It includes functionalities for creating an array of structures, reading data from a file, conducting sentiment analysis, and freeing memory..
\begin{lstlisting}[label={list:first},caption=functions.h]
#ifndef FUNCTIONS_H
#define FUNCTIONS_H

// Structure to store the words, their scores, standard deviation and SIS array
struct words
{
    char *word;
    float score;
    float SD;
    int SIS_array[10];
};

// Function to make an array of struct
void make_aos(int *wordlist_size, struct words **wordlist);

// Function to read data from the file and store it in an array of struct
void get_data(char *file_name,int *wordlist_size, struct words **wordlist);

// Funstion to free memory allocated for the array of struct
void free_data(struct words *wordlist);

// Function to perform Sentiment Analysis using VADER
void sentiment_analysis(struct words *wordlist, char *validation_file);

#endif
\end{lstlisting}

\section*{2) functions.c}
The functions.c file contains a collection of functions related to sentiment analysis. Firstly, it imports the necessary libraries, including the 'functions.h' header file which contains the function declarations.
\begin{lstlisting}[label={list:second},caption=function.c]
#include <stdio.h>
#include <stdlib.h>
#include <string.h>
#include <ctype.h>
#include "functions.h"
\end{lstlisting}

\section*{3) make\_aos() function}
This function dynamically allocates memory for an array of structures based on the provided size (wordlist\_size). It correctly assigns the allocated memory to the pointer *wordlist.
\begin{lstlisting}[label={list:third},caption=make\_aos() function]
// Function to make an array of structures
void make_aos(int *wordlist_size, struct words **wordlist)
{   
    // Allocate memory for the array of structures
    *wordlist = malloc(sizeof(struct words) * (*wordlist_size));
    
    // Check if memory is allocated
    if (*wordlist == NULL)
    {
        printf("Error: Unable to allocate memory\n");
        exit(1);
    }
}
\end{lstlisting}

\section*{4) get\_data() function}
The function reads data from the specified file and stores it in the array of structures pointed to by *wordlist. It handles dynamically resizing the array when it becomes full. The data is tokenized using strtok() and stored appropriately in the structure fields. It correctly deallocates the memory if an error occurs during memory reallocation.
\begin{lstlisting}[label={list:fourth},caption=get\_data() function]
// Function to read data from the file and store it in an array of structures
void get_data(char *file_name,int *wordlist_size, struct words **wordlist)
{
    // Open the file
    FILE *file = fopen(file_name, "r");

    // Check if the file is opened
    if (file == NULL)
    {
        printf("Error: Unable to open file %s\n", file_name);
        exit(1);
    }

    // array to store one line of the file 
    char line[200];
    // counter to keep track of the number of words(lines)
    int c = 0;

    // Read the file line by line
    while (fgets(line, sizeof(line), file) != NULL)
    {
        // Check if the array is full
        if (c >= *wordlist_size)
        {
            // Increase the size of the array by 1000
            *wordlist_size += 1000;
            // Reallocate memory for the array
            *wordlist = realloc(*wordlist, sizeof(struct words) * (*wordlist_size));

            // Check if memory is allocated
            if (*wordlist == NULL)
            {
                printf("Error: Unable to allocate memory\n");
                exit(1);
            }

        }

        // Structure to store the data
        struct words new_word;

        // Tokenize the line
        char *token = strtok(line, " \t\n");

        // Store the data in the structure
        new_word.word = strdup(token);

        token = strtok(NULL, " \t\n");
        new_word.score = atof(token);

        token = strtok(NULL, " \t\n");
        new_word.SD = atof(token);

        // Tokenize the SIS array
        char *arr_token = strtok(NULL, ", []\t\n");
        int i = 0;

        // Store the SIS array in the structure
        while (arr_token != NULL)
        {
            new_word.SIS_array[i] = atoi(arr_token);
            arr_token = strtok(NULL, ", []\t\n");
            i++;
        }

        // Store the structure in the array
        (*wordlist)[c] = new_word;
        c++;

    }
    // Close the file
    fclose(file);
}

\end{lstlisting}

\section*{5) free\_data() function}
This function correctly frees the memory allocated for each word in the array of structures and then frees the memory for the array itself.
\begin{lstlisting}[label={list:fifth},caption=free\_data() function]
// Function to free memory allocated for the array of structures
void free_data(struct words *wordlist)
{
    // Free the memory allocated for each word
    for (int i = 0; wordlist[i].word != NULL; i++) {
        free(wordlist[i].word);
    }
    // Free the memory allocated for the array
    free(wordlist);
}
\end{lstlisting}

\section*{6) sentiment\_analysis() function}
This function performs sentiment analysis on the text lines read from the validation file. It correctly opens the validation file and reads lines from it. Sentiment analysis is done by comparing each word in the line with the words stored in the array of structures (wordlist). It accumulates the total score for each line based on the words' scores.
\begin{lstlisting}[label={list:sixth},caption=sentiment\_analysis() function]
// Function to perform Sentiment Analysis using VADER
void sentiment_analysis(struct words *wordlist, char *validation_file)
{
    // Open the validation file
    FILE *val_file = fopen(validation_file, "r");

    // Check if the file is opened
    if (val_file == NULL)
    {
        printf("Error: Unable to open file %s\n", validation_file);
        exit(1);
    }

    // Read the file line by line
    char line[200];

    // Perform the sentiment analysis
    while (fgets(line, sizeof(line), val_file) != NULL)
    {
        // Variables to store the total score and the number of words in each line
        float total_score = 0;
        int c = 0;
        
        // Copy the line to another variable
        char line_copy[200];
        strcpy(line_copy, line);

        // Remove the newline character put by fgets
        line_copy[strlen(line_copy) - 1] = '\0';

        // Convert the line to lowercase
        for (int i = 0; line[i] != '\0'; i++)
        {
            line[i] = tolower(line[i]);
        }
        
        // Tokenize the line
        char *token = strtok(line, " \n\t!,.");

        // Calculate the total score
        while (token != NULL)
        {
            // iterate through the wordlist
            for (int i = 0; wordlist[i].word != NULL; i++)
            {
                // Compare the token with the word in the wordlist
                if (strcmp(token, wordlist[i].word) == 0)
                {
                    // Add the score of the word to the total score
                    total_score += wordlist[i].score;
                }
            }
            c++;
            // Get the next word
            token = strtok(NULL, " \n\t!,.");
        }
        
        // Check if the number of words is greater than 0
        if (c > 0)
        {
            // Calculate the average score
            total_score = total_score / c;
            // Print the line and the total score (%-100s is used to print the string in 100 characters width to the left and %10.2f is used to print the float with 2 decimal places)
            printf("%-100s %10.2f", line_copy, total_score);
            printf("\n");
        }

    }
    
    // Close the file
    fclose(val_file);
}
\end{lstlisting}

\section*{7) main() function}
The main() function serves as the entry point of the program. It orchestrates the flow of the program, from argument handling to memory management and execution of sentiment analysis, ensuring proper functionality and resource utilization.
\begin{lstlisting}[label={list:seventh},caption=main() function]
int main(int argc, char *argv[])
{   
    // Check if the number of arguments is correct
    if (argc != 3)
    {
        // Print the usage of the program
        printf("Usage: %s <SA_dictionary> <Validation_file>\n", argv[0]);
        return 1;
    }

    // Get the file name and the validation file
    char *file_name = argv[1];
    char *validation_file = argv[2];

    // initialize the wordlist
    int wordslist_size = 5000;
    struct words *wordlist = NULL;
    
    // Make the array of structures
    make_aos(&wordslist_size, &wordlist);

    // Get the data from the file and store it in the wordlist
    get_data(file_name, &wordslist_size, &wordlist);

    // Print the header
    printf("              String sample                                                                               Score\n");
    printf("----------------------------------------------------------------------------------------------------------------\n");

    // Perform the sentiment analysis
    sentiment_analysis(wordlist, validation_file);

    // Free the data
    free_data(wordlist);

    return 0;
}
\end{lstlisting}

\section*{8) Makefile}
\begin{lstlisting}[label={list:eighth},caption=Makefile]
CC = gcc
CFLAGS = -Wall -Wextra 
EXECUTABLE = mySA
SRC = sa.c
OBJSRC = functions.c
OBJ = functions.o
HLP = functions.h

$(EXECUTABLE): $(SRC) $(OBJ)
	$(CC) $(CFLAGS) -o $(EXECUTABLE) $(SRC) $(OBJ)
$(OBJ): $(OBJSRC) $(HLP)
	$(CC) -c $(OBJSRC) $(CFLAGS) -o $(OBJ) 
clean:
	rm -f mySA functions.o
\end{lstlisting}

\section*{9) Usage}
\begin{itemize}
    \item Compile the program using 'make' command.
    \item Run the program using the following command.
    \begin{lstlisting}[label={list:ninth},caption=Run]
    ./SA <dictionary\_name> <validation\_file>
    \end{lstlisting}
    \item The dictionary file should be of the following format
    \begin{lstlisting}[label={list:tenth},caption=Dictionary format]
     <word> <score> <standard_deviation> [<SIS_array>]
    \end{lstlisting}
    \item Output:
    \begin{lstlisting}[label={list:eleventh},caption=Output Example]
     ./mySA vader_lexicon.txt validation.txt
    \end{lstlisting}
    \begin{figure}
    \centering
    \includegraphics[width=1\linewidth]{mySA.jpg}
    \caption{Enter Caption}
    \label{fig:enter-label}
    \end{figure}
\end{itemize}


\section*{9) Appendix}

\begin{lstlisting}[label={list:twelvth},caption=Appendix]
// Import the necessary libraries
#include <stdio.h>
#include <stdlib.h>
#include <string.h>
#include <ctype.h>
#include "functions.h"


int main(int argc, char *argv[])
{   
    // Check if the number of arguments is correct
    if (argc != 3)
    {
        // Print the usage of the program
        printf("Usage: %s <SA_dictionary> <Validation_file>\n", argv[0]);
        return 1;
    }

    // Get the file name and the validation file
    char *file_name = argv[1];
    char *validation_file = argv[2];

    // initialize the wordlist
    int wordslist_size = 5000;
    struct words *wordlist = NULL;
    
    // Make the array of structures
    make_aos(&wordslist_size, &wordlist);

    // Get the data from the file and store it in the wordlist
    get_data(file_name, &wordslist_size, &wordlist);

    // Print the header
    printf("              String sample                                                                               Score\n");
    printf("----------------------------------------------------------------------------------------------------------------\n");

    // Perform the sentiment analysis
    sentiment_analysis(wordlist, validation_file);

    // Free the data
    free_data(wordlist);

    return 0;
}

#include <stdio.h>
#include <stdlib.h>
#include <string.h>
#include <ctype.h>
#include "functions.h"

// Function to make an array of structures
void make_aos(int *wordlist_size, struct words **wordlist)
{   
    // Allocate memory for the array of structures
    *wordlist = malloc(sizeof(struct words) * (*wordlist_size));
    
    // Check if memory is allocated
    if (*wordlist == NULL)
    {
        printf("Error: Unable to allocate memory\n");
        exit(1);
    }
}

// Function to read data from the file and store it in an array of structures
void get_data(char *file_name,int *wordlist_size, struct words **wordlist)
{
    // Open the file
    FILE *file = fopen(file_name, "r");

    // Check if the file is opened
    if (file == NULL)
    {
        printf("Error: Unable to open file %s\n", file_name);
        exit(1);
    }

    // array to store one line of the file 
    char line[200];
    // counter to keep track of the number of words(lines)
    int c = 0;

    // Read the file line by line
    while (fgets(line, sizeof(line), file) != NULL)
    {
        // Check if the array is full
        if (c >= *wordlist_size)
        {
            // Increase the size of the array by 1000
            *wordlist_size += 1000;
            // Reallocate memory for the array
            *wordlist = realloc(*wordlist, sizeof(struct words) * (*wordlist_size));

            // Check if memory is allocated
            if (*wordlist == NULL)
            {
                printf("Error: Unable to allocate memory\n");
                exit(1);
            }

        }

        // Structure to store the data
        struct words new_word;

        // Tokenize the line
        char *token = strtok(line, " \t\n");

        // Store the data in the structure
        new_word.word = strdup(token);

        token = strtok(NULL, " \t\n");
        new_word.score = atof(token);

        token = strtok(NULL, " \t\n");
        new_word.SD = atof(token);

        // Tokenize the SIS array
        char *arr_token = strtok(NULL, ", []\t\n");
        int i = 0;

        // Store the SIS array in the structure
        while (arr_token != NULL)
        {
            new_word.SIS_array[i] = atoi(arr_token);
            arr_token = strtok(NULL, ", []\t\n");
            i++;
        }

        // Store the structure in the array
        (*wordlist)[c] = new_word;
        c++;

    }
    // Close the file
    fclose(file);
}

// Function to free memory allocated for the array of structures
void free_data(struct words *wordlist)
{
    // Free the memory allocated for each word
    for (int i = 0; wordlist[i].word != NULL; i++) {
        free(wordlist[i].word);
    }
    // Free the memory allocated for the array
    free(wordlist);
}

// Function to perform Sentiment Analysis using VADER
void sentiment_analysis(struct words *wordlist, char *validation_file)
{
    // Open the validation file
    FILE *val_file = fopen(validation_file, "r");

    // Check if the file is opened
    if (val_file == NULL)
    {
        printf("Error: Unable to open file %s\n", validation_file);
        exit(1);
    }

    // Read the file line by line
    char line[200];

    // Perform the sentiment analysis
    while (fgets(line, sizeof(line), val_file) != NULL)
    {
        // Variables to store the total score and the number of words in each line
        float total_score = 0;
        int c = 0;
        
        // Copy the line to another variable
        char line_copy[200];
        strcpy(line_copy, line);

        // Remove the newline character put by fgets
        line_copy[strlen(line_copy) - 1] = '\0';

        // Convert the line to lowercase
        for (int i = 0; line[i] != '\0'; i++)
        {
            line[i] = tolower(line[i]);
        }
        
        // Tokenize the line
        char *token = strtok(line, " \n\t!,.");

        // Calculate the total score
        while (token != NULL)
        {
            // iterate through the wordlist
            for (int i = 0; wordlist[i].word != NULL; i++)
            {
                // Compare the token with the word in the wordlist
                if (strcmp(token, wordlist[i].word) == 0)
                {
                    // Add the score of the word to the total score
                    total_score += wordlist[i].score;
                }
            }
            c++;
            // Get the next word
            token = strtok(NULL, " \n\t!,.");
        }
        
        // Check if the number of words is greater than 0
        if (c > 0)
        {
            // Calculate the average score
            total_score = total_score / c;
            // Print the line and the total score (%-100s is used to print the string in 100 characters width to the left and %10.2f is used to print the float with 2 decimal places)
            printf("%-100s %10.2f", line_copy, total_score);
            printf("\n");
        }

    }
    
    // Close the file
    fclose(val_file);
}
\end{lstlisting}



\section*{10) References}

\begin{itemize}
    \item Large-language models 
\end{itemize}

\end{document}
